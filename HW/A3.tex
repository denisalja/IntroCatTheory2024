\documentclass[11pt]{amsart}
\usepackage{amssymb, amstext, amsthm, amscd, amsmath,color}
\usepackage{graphicx} 
\usepackage[margin = 1 in]{geometry}
\usepackage{tikz-cd}
\usepackage{cancel}
\usepackage{thmtools}
\usepackage{mathtools}
\usepackage{stmaryrd}
\usepackage{comment}
\usepackage{enumitem}
\usepackage{xargs}
\usetikzlibrary{
  knots,
  hobby,
  decorations.pathreplacing,
  shapes.geometric,
  calc,
  decorations.pathmorphing,
  shapes,
  decorations.markings
  }
\usepackage[hidelinks]{hyperref}
\usepackage{tensor}
\usepackage{mathpartir}
\usepackage[textsize=scriptsize]{todonotes}
\newcommand{\deni}[1]{\todo[color=blue!25]{{\bf Deni:} #1}}
\newcommand{\bigdeni}[1]{\todo[inline,color=blue!25]{{\bf Deni:} #1}}
%Hyperlinks
\hypersetup{
colorlinks=true,
linktoc=all
}
\theoremstyle{plain}
\newtheorem{thm}{Theorem}[]
\newtheorem{cor}[thm]{Corollary}
\newtheorem{prop}[thm]{Proposition}
\newtheorem{lem}[thm]{Lemma}
\newtheorem*{propp}{Proposition}
\newtheorem*{corr}{Corollary}
%   Theorem style with roman text
%   numbered within section
\theoremstyle{definition}
\newtheorem{rem}[thm]{Remark}
\newtheorem{prob}{Problem}
\newtheorem{rems}[thm]{Remarks}
\newtheorem{defn}[thm]{Definition}
\newtheorem*{defn*}{Definition}
\newtheorem*{rem*}{Remark}
\newtheorem*{eg*}{Example}
\newtheorem*{neg*}{Non-Example}
\newtheorem*{egs*}{Examples}
\renewcommand*{\proofname}{Solution}
\newcommand{\fty}{\infty}
\newcommand{\lam}{\lambda}
\newcommand{\mR}{{\mathbb R}}
\newcommand{\mZ}{{\mathbb Z}}
\newcommand{\mN}{{\mathbb N}}
\newcommand{\vep}{\varepsilon}
\newcommand{\mD}{{\mathbb D}}
\newcommand{\mC}{{\mathbb C}}
\newcommand{\mE}{{\mathbb E}}
\newcommand{\mP}{{\mathbb P}}
\newcommand{\mS}{{\mathbb S}}
\newcommand{\mT}{{\mathbb T}}
\newcommand{\mB}{{\mathbb B}}
\newcommand{\mF}{{\mathbb F}}
\newcommand{\mK}{{\mathbb K}}
\newcommand{\mL}{{\mathbb L}}
\newcommand{\mA}{{\mathbbl A}}
\newcommand{\cZ}{{\mathcal Z}}
\newcommand{\cA}{{\mathcal A}}
\newcommand{\cB}{{\mathcal B}}
\newcommand{\cP}{{\mathcal P}}
\newcommand{\cC}{{\mathcal C}}
\newcommand{\cD}{{\mathcal D}}
\newcommand{\cU}{{\mathcal U}}
\newcommand{\cE}{{\mathcal E}}
\newcommand{\cM}{{\mathcal M}}
\newcommand{\cV}{{\mathcal V}}
\newcommand{\cY}{{\mathcal Y}}
\newcommand{\cJ}{{\mathcal J}}
\DeclareMathOperator{\Aut}{Aut}
\DeclareMathOperator{\Ext}{Ext}
\DeclareMathOperator{\rk}{rk}
\DeclareMathOperator{\id}{\mathsf{id}}
\newcommand{\Nat}{\mathrm{Nat}}
\newcommand{\cov}{\mathrm{cov}}
\newcommand{\cod}{\mathrm{cod}}
\newcommand{\dom}{\mathrm{dom}}
\newcommand{\Hom}{\text{Hom}}
%
\newcommand{\im}{\text{im}}

\newcommand{\Rel}{\mathbf{Rel}}
\newcommand{\PAb}{\mathbf{PAb}}
\newcommand{\PSh}{\mathbf{PSh}}
\newcommand{\Sh}{\mathbf{Sh}}
\newcommand{\Top}{\mathbf{Top}}
\newcommand{\Cat}{\mathbf{Cat}}
\newcommand{\Ch}{\mathbf{Ch}}
\newcommand{\bA}{\mathbf{A}}
\newcommand{\bB}{\mathbf{B}}
\newcommand{\bD}{\mathbf{D}}
\newcommand{\bP}{\mathbf{P}}
\newcommand{\Sp}{{\mathbf{Sp}}}
\newcommand{\Set}{{\mathbf{Set}}}
\newcommand{\Grp}{\mathbf{Grp}}
\newcommand{\Ab}{\mathbf{Ab}}
\newcommand{\Par}{\mathbf{Par}}
\newcommand{\Ring}{\mathbf{Ring}}
\newcommand{\fkm}{\mathfrak{m}}
\newcommand{\bG}{\mathbf{G}}
\newcommand{\bH}{\mathbf{H}}
\newcommand{\bbA}{{\mathbb{A}}}
\newcommand{\mbJ}{{\mathbf{J}}}
\newcommand{\mbX}{{\mathbf {X}}}
\newcommand{\mbY}{{\mathbf {Y}}}
\newcommand{\mbZ}{{\mathbf {Z}}}
\newcommand{\noi}{{\noindent}}
\newcommand{\mQ}{{\mathbb Q}}
\newcommand*\quot[2]{{^{\textstyle #1}\big/_{\textstyle #2}}}
\newcommand{\cO}{\mathcal{O}}
 %For Adjuncitons
 \tikzset{%
    symbol/.style={%
        draw=none,
        every to/.append style={%
            edge node={node [sloped, allow upside down, auto=false]{$#1$}}}
    }
}
%left adjoint on top adjunction
\newcommand*\ladj[4]{\begin{tikzcd}[column sep = large, row sep = large, ampersand replacement=\&]
{#1} \ar[r,bend left,"{#2}",
{name=A, below}] \&  {#3} \ar[l,bend left,"{#4}",
{name=B,above}] \ar[from=A, to=B, symbol={\dashv}]
\end{tikzcd} }
%right adjoint on top adjunction
\newcommand*\radj[4]{\begin{tikzcd}[column sep = large, row sep = large, ampersand replacement=\&]
{#1} \ar[r,bend left,"{#2}",
{name=A, below}] \&  {#3} \ar[l,bend left,"{#4}",
{name=B,above}] \ar[from=B, to=A, symbol={\dashv}]
\end{tikzcd} }
%left adjoint on top adjunction
\newcommand*\ladjvcomp[5]{\begin{tikzcd}[column sep = huge, row sep = huge, ampersand replacement=\&]
{#1} \ar[r,bend left = 50,"{#2}",
{name=A, below}] 
\ar[from = r,"{#3}", 
{name=B,above},{name=B',bottom}]
 \ar[r,bend right=50,"{#4}"',
{name=C,below}]
\& {#5}  \ar[from=A, to=B, symbol={\dashv}]
\ar[from=B, to=C, symbol={\dashv}]
\end{tikzcd} }
%HOM-ALG SHORTCUTS
%sequence tikzcd arrows: \seqtkz{domain}{name}{codomain = domain}{name}{codomain}
\newcommand*\seqtkz[5]{{\begin{tikzcd}[ampersand replacement=\&] 
 \cdots \rar \& { #1 } \rar["{#2}"] \&  { #3 } \rar["{#4}"] \&  { #5 } \rar \& \cdots \end{tikzcd}}}
%short exact sequence: \ses{ob1}{map1}{ob2 }{map2}{ob3}
\newcommand*\ses[5]{{\begin{tikzcd}[ampersand replacement=\&] 
 0 \rar \& { #1 } \rar["{#2}"] \&  { #3 } \rar["{#4}"] \&  { #5 } \rar \& 0 \end{tikzcd}}}
%induced les cohomology: \lesco{space}{A}{B}{C}{varphi}{psi}
\newcommandx*\lescoh[6]{{\begin{tikzcd}[ampersand replacement=\&] 
 0 \rar \& H^0({#1};{#2} )  \rar["{#5}^*"] \&  H^0({#1};{#3})  \rar["{#6}^*"] \&  H^0({#1};{#4} ) \rar["\delta^0"] \& \cdots \
  \cdots \rar["\delta^n"] \& H^n({#1};{#2})  \rar["{#5}^*"] \& H^n({#1};{#3})  \rar["{#6}^*"] \&  H^n({#1};{#4})  \rar \& \cdots  
  \end{tikzcd}}}
%Injective Resolution: \artkz{domain}{name}{codomain}
\newcommand*\injresl[3]{{\begin{tikzcd}[ampersand replacement=\&] 
{ #1 } \rar[tail, "{#2}"] \&  { #3 }^\bullet \end{tikzcd}}}
%Injective Resolution Open ended: \injreslo{domain}{name}{codomain}
\newcommand*\injreslo[5]{{\begin{tikzcd}[ampersand replacement=\&] 
0 \rar[] \& { #1 } \rar[tail, "{#2}"] \&  { #3 } \rar["{#4}"] \& {#5} \rar \& \cdots  \end{tikzcd}}}
%Proj Resolution: \artkz{domain}{name}{codomain}
\newcommand*\projresl[3]{{\begin{tikzcd}[ampersand replacement=\&] 
{ #1 } \rar["{#2}"] \&  { #3 }_\bullet \end{tikzcd}}}
%etale spce functor
\newcommand{\Et}{\acute{E}t}
%germs
\newcommand*\germ[2]{\mathrm{germ_{#1}({#2})}}
%opens
\newcommand*\Op[1]{\mathbf{Open}({#1})}
\newcommand*\Opo[1]{\mathbf{Open}({#1})^{op}}
%support
\newcommand{\supp}{\mathrm{supp}}
\newcommand{\ab}{\mathrm{Ab}}
%big roomy quotient
\newcommand{\nat}[6][large]{%
 \begin{tikzcd}
 [ampersand replacement = \&, column sep=#1]
  #2\ar[bend left=40,
{name=U}]{r}{#4}
  \ar[bend right=40,',
{name=D}]{r}{#5}
  \& #3 \ar[shorten <=10pt,shorten >=10pt,Rightarrow,from=U,to=D]{d}{~#6}
  \end{tikzcd}
}
%pullback
\newsavebox{\pullback}
\sbox\pullback{%
\begin{tikzpicture}%
\draw (0,0) -- (1ex,0ex);%
\draw (1ex,0ex) -- (1ex,1ex);%
\end{tikzpicture}}
%pullback comand: \arrow[dr, phantom, "\usebox\pullback" , very near start, color=black]
%pushout
\newsavebox{\pushout}
\sbox\pushout{%
\begin{tikzpicture}%
\draw (0,0) -- (0ex,1ex);%
\draw (0ex,1ex) -- (1ex,1ex);%
\end{tikzpicture}}
%pushout comand: \arrow[ul, phantom, "\usebox\pushout" , very near start, color=black]
%Squiggly Arrow Guy: 
\newcounter{sarrow}
\newcommand\xrsquigarrow[1]{%
\stepcounter{sarrow}%
\begin{tikzpicture}[decoration=snake]
\node (\thesarrow) {\strut#1};
\draw[->,decorate] (\thesarrow.south west) -- (\thesarrow.south east);
\end{tikzpicture}%
}
\title{Assignment 3}
\author{Deni Salja}
\begin{document}
\maketitle

\begin{prob}
A functor that doesn't preserve mono's. 
\end{prob}
\begin{proof}
Consider the functor $F$ from the walking arrow category $\mathbf{2}$ 

\[\begin{tikzcd}
	a \rar["\varphi"] & b
\end{tikzcd}\]

\noi to the two-pronged fork category $\mathbf{2Frk}$ 

\[\begin{tikzcd}
	x & y & z
	\arrow["f", from=1-2, to=1-3]
	\arrow["g", shift left=2, from=1-1, to=1-2]
	\arrow["h"', shift right=2, from=1-1, to=1-2]
\end{tikzcd}\]

\noi defined by picking out the handle of the fork. 

\[ F(\varphi) = f. \]

\noi The map $\varphi \in \mathbf{2}$ is vacuously monic but $f$ is not: $f \circ g = f \circ h$ are equal by definition but $g \neq h$. 
\end{proof}

\begin{prob}
A functor that doesn't reflect mono's. 
\end{prob}
\begin{proof}
The functor $F : \mathbf{2} \to \mathbf{2Frk}$ has a retraction $G: \mathbf{2Frk} \to \mathbf{2}$ defined by

\[ G(g) = 1_a = G(h) , \quad G(f) = \varphi\]

\noi As mentioned above, $G(f) = \varphi$ is vacuously monic in $\mathbf{2}$ but $f$ is not monic in $\mathbf{2Frk}$. 
\end{proof}

\begin{prob}
Faithful functors reflect mono's
\end{prob}
\begin{proof}
Let $F : \cC \to \cD$ be faithful and suppose $F(f)$ is monic in $\cD$. Further suppose $f \circ g = f \circ h$ in $\cC$. Applying $F$ to these composites we see

\[ F(f) \circ F(g) = F(f \circ g) = F(f \circ h) = F(f) \circ F(h).\]

\noi Since $F(f)$ is monic $F(g) = F(h)$ and since $F$ is faithful we have $g = h$. This shows $f$ is monic in $\cC$ and concludes the proof. 
\end{proof}

\begin{prob}
If $g \circ f$ is monic then $f$ is monic and if $g \circ f$ is epic then $g$ is epic. 
\end{prob}
\begin{proof}
Epi's and mono's are dual so these two statements (along with their) are formally dual and it suffices to prove the first one. \medskip 

Assume $g \circ f$ is monic and suppose $f \circ h = f \circ k$. Post-composing with $g$ and omitting brackets for composition we see
\[ g \circ f \circ h = g \circ f \circ k \]

\noi Since $g \circ f$ is monic we get $h =k$. This implies $f$ is monic. 
\end{proof}

\begin{prob}
Show that every functor preserves sections and retractions. 
\end{prob}
\begin{proof}
Sections and retractions are formally dual so it suffices to prove every functor preserves sections. \medskip 

Let $F : \cC \to \cD$ be a functor and $s$ a section in $\cC_1$, ie.~there exists composable $r \in \cC_1$ such that $r \circ s = 1$ is the identity on the domain of $s$. Functors preserve composition and identities so we have

\[ F(r) \circ F(s) = F(r \circ s) = F(1) = 1.\]

\noi This shows $F(s)$ is a section (of $F(r)$) in $\cD$ and so $F$ preserves sections.
\end{proof}

\begin{prob}
Show that the unique functor $\cC \to \mathbf{1}$ is faithful if and only if $\cC$ is a preorder category. 
\end{prob}
\begin{proof}
Suppose the functor $\cC \to \mathbf{1}$ is faithful. Then for each $A , B \in \cC$ the induced map sending all arrows to the identity

\[ \label{homsets}\cC(A,B) \to \{ 1 \} \tag{$\star$}\]

\noi is injective. More explicitly for any $f, g : A \to B$ in $\cC$ we must have $f = g$. This implies ever hom-set of $\cC$ has at most one arrow, ie.~$\cC$ is a preorder category. \bigskip 

On the other hand if $\cC$ is a preorder category then it's hom-sets have at most one arrow. This means the map on hom-sets (\ref{homsets}) induced by the unique functor $\cC \to \mathbf{1}$ is injective and therefore the functor is faithful.  
\end{proof}

\begin{prob}
Problem 1.3.ix in Riehl's book (Cat's in Context). The commutator subgroup, the automorphism group, and the center of a group are all groups we can assign to a group. \medskip

These are all functorial on the discrete category of groups in a trivial way. Is it still functorial on the category of groups with isomorphisms between them? On the category of groups with epimorphisms/monomorphisms between them? On the category of groups and all group homomorphisms. 
\end{prob}
\begin{proof}\
\begin{enumerate}[label=(\alph*)]
\item Assigning each group $G \in \Grp_0$ its commutator subgroup $C(G)$ is functorial on all of $\Grp$ (and therefore any of its subcategories). For $\varphi : G \to H$ we can define $C(\varphi)$ on the generators of $C(G)$ by applying $\varphi$

\[ 
C(\varphi)(ghg^{-1}h^{-1}) 
:= \varphi(ghg^{-1}h^{-1}) 
= \varphi(g) \varphi(h) \varphi(g)^{-1} \varphi(h)^{-1} 
\]

\noi The second equality comes from the fact that $\varphi$ is a group homomorphism. Similarly $C(\varphi)$ is a group homomorphism because $\varphi$ is. Identities are clearly preserved


\[ C(1_G)(ghg^{-1}h^{-1}) = 1_G(ghg^{-1}h^{-1}) =ghg^{-1}h^{-1} = 1_{C(G)} (ghg^{-1}h^{-1})\]

\noi and so is composition 

\begin{align*} 
C(\varphi \circ \psi) (ghg^{-1}h^{-1}) 
&= \varphi \circ \psi (ghg^{-1}h^{-1}) \\
&=  (\varphi \circ \psi (g)) (\varphi \circ \psi(h)) (\varphi \circ \psi(g)^{-1})(\varphi \circ \psi(h)^{-1})\\ 
&= C(\varphi) \left(\psi (g) \psi(h) \psi(g)^{-1} \psi(h)^{-1} \right)\\ 
&= C(\varphi) \circ C(\psi) ( ghg^{-1}h^{-1}).
\end{align*}\bigskip

\item Assigning each group $G \in \Grp_0$ to its center $Z(G)$ is functorial on any subcategory of group whose arrows are epimorphisms, including the subcategory of isomorphisms.  \bigskip 

Let $\varphi : G \to H$ be a group homomorphism. Then $g \in Z(G)$ whenver it commutes with all $h \in G$. Since $\varphi$ is a group homomorphism it preserves this commutativity relation on the image: if $gh = hg$ for all $h \in G$ then 

\[ \varphi(g) \varphi(h) = \varphi(gh) = \varphi(hg) = \varphi(h) \varphi(g) \] 

\noi shows $\varphi (g) \in Z( \im \varphi ) \supseteq Z(H)$. The epimorphisms in $\Grp$ are precisely the surjective group homomorphisms so 

\[ Z \varphi : Z(G) \to Z(H) \quad ; \qquad Z \varphi(g) = \varphi(g)\]

\noi is well-defined whenever $\varphi$ is an epimorphism. Identities are epi's and they're clearly preserved:

\[ Z(1_G)(g) = 1_G(g) = g = 1_{Z(G)}\]

\noi Epimorphisms are stable under composition and 

\[ Z(\varphi \circ \psi) (g) = \varphi \circ \psi (g) = \varphi \left( Z(\psi) (g)\right) = \left(Z(\varphi) \circ Z(\psi)\right) (g) \]

\noi shows they're preserved. It follows that $Z(-)$ is functorial on any subcategory of groups whose arrows are all epimorphisms. \bigskip 

\item The automorphism group assignment, $\Aut$, extends to the subcategory of isomorphisms in $\Grp$. If $\varphi : A \to B$ is a group isomorphism define $\Aut(\varphi)$ by conjugation with $\varphi$:

\[ \Aut(\varphi) : \Aut(A) \to \Aut(B) \quad ; \qquad \Aut(\varphi) (\alpha) = \varphi^{-1} \circ \alpha \circ \varphi \]

\noi This preserves identities: for all $A \in \Grp_0$ 

\[ \Aut(1_A) (\alpha) = 1_A^{-1} \circ \alpha \circ 1_A = \alpha = 1_{\Aut(A)}(\alpha )\] 

\noi It also preserves composition: for all composable isomorphisms of groups $\varphi$ and $\psi$

\begin{align*} 
\Aut(\varphi \circ \psi )(\alpha) 
&= (\varphi \circ \psi)^{-1} \circ \alpha \circ (\varphi \circ \psi) \\
&=  \psi^{-1} \circ \varphi^{-1} \circ \alpha \circ \varphi \circ \psi \\
&=   \psi^{-1} \circ \Aut(\varphi) (\alpha) \circ \psi \\
&= \Aut(\psi) \circ \Aut(\varphi) (\alpha).
\end{align*} 
\end{enumerate}
\end{proof}

\end{document}